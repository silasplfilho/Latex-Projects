% ---
% RESUMOS
% ---

% RESUMO em português
\setlength{\absparsep}{18pt} % ajusta o espaçamento dos parágrafos do resumo
\begin{resumo}
% A inovação é um fator condicionante para o desenvolvimento organizacional, geração de riquezas e liderança de mercado. No cenário de mudanças recentes, nota-se um encurtamento do ciclo de vida dos produtos, cada vez mais condicionado aos desejos dos clientes e de tendências percebidas no mercado, exigindo respostas rápidas das organizações. Diante desse quadro, as startups vêm ganhando papel de destaque em função da flexibilidade do modelo de negócio, baixos investimentos iniciais e maior dinamicidade nos processos de transferência tecnológica. Como forma de viabilizar tais negócios, um número crescente de instituições de apoio tem se estabelecido para orientar empreendimentos nos seus primeiros dias. São investidores e capitalistas de risco, incubadoras e aceleradoras, além de órgãos governamentais de suporte – atores que compõem os ecossistemas de startups, provendo infraestrutura, espaço de mercado, parcerias consolidadas e processos estabelecidos. Essa comunidade empreendedora tão plural contribui em diferentes perspectivas para os novos empreendimentos, contudo traz uma série de desafios relacionados a convergência de interesses na rede. É fundamental para o ecossistema compreender quais são as competências, oportunidades e afinidades entre os membros para assegurar a harmonia e garantir a prosperidade do meio. 
%O presente trabalho apresenta um método para auxiliar no mapeamento dos recursos disponíveis, papel dos participantes, relações e comportamentos da comunidade baseado em métricas de  análise de redes sociais. Acredita-se que as respostas obtidas através deste estudo auxiliarão nos processos de tomada de decisões e na melhor compreensão das características e recursos disponíveis na rede.

 \textbf{Palavras-chaves}: Informática na Saúde Mental. Depressão. Análise de Redes Sociais. 
\end{resumo}

% ABSTRACT in english
\begin{resumo}[Abstract]
 \begin{otherlanguage*}{english}
% Innovation is a fundamental aspect of organizational development, wealth generation and market leadership. It has been noticed the shortening of the products lifecycle, which are much more conditioned by customer wishes and perceived market trends. It requires rapid responses from enterprises. Given this background, the lean startups have been gaining a prominent role due to the flexibility of the business model, low initial costs and greater dynamism in the technology diffusion. As a way to enable such businesses, a crescent number of support institutions have established to guide ventures in their early stages. They are made up of investors and venture capitalists, incubators and accelerators, as well as support government agencies - actors that compose the startup ecosystems, providing infrastructure, market space, consolidated partnerships and established processes. This heterogeneous entrepreneurial community contributes in different perspectives for these new firms. However it brings a series of challenges related to the convergence of interests in the network. It is critical for the ecosystem to understand the competencies, opportunities and affinities among members to take harmony and ensure the prosperity of the environment. This work presents a method to aid the identification and characterization of partnerships ties based on metrics of social network analysis. It is believed that the answers obtained through this study will help in the decision-making processes and the better understanding of the characteristics and resources available in the startup ecosystem.

   \vspace{\onelineskip}
 
   \noindent 
   \textbf{Keywords}: Mental Health Informatics. Depression. Social Network Analysis.
 \end{otherlanguage*}
\end{resumo}