% This is "sig-alternate.tex" V2.1 April 2013
% This file should be compiled with V2.5 of "sig-alternate.cls" May 2012
%
% This example file demonstrates the use of the 'sig-alternate.cls'
% V2.5 LaTeX2e document class file. It is for those submitting
% articles to ACM Conference Proceedings WHO DO NOT WISH TO
% STRICTLY ADHERE TO THE SIGS (PUBS-BOARD-ENDORSED) STYLE.
% The 'sig-alternate.cls' file will produce a similar-looking,
% albeit, 'tighter' paper resulting in, invariably, fewer pages.
%
% ----------------------------------------------------------------------------------------------------------------
% This .tex file (and associated .cls V2.5) produces:
%       1) The Permission Statement
%       2) The Conference (location) Info information
%       3) The Copyright Line with ACM data
%       4) NO page numbers
%
% as against the acm_proc_article-sp.cls file which
% DOES NOT produce 1) thru' 3) above.
%
% Using 'sig-alternate.cls' you have control, however, from within
% the source .tex file, over both the CopyrightYear
% (defaulted to 200X) and the ACM Copyright Data
% (defaulted to X-XXXXX-XX-X/XX/XX).
% e.g.
% \CopyrightYear{2007} will cause 2007 to appear in the copyright line.
% \crdata{0-12345-67-8/90/12} will cause 0-12345-67-8/90/12 to appear in the copyright line.
%
% ---------------------------------------------------------------------------------------------------------------
% This .tex source is an example which *does* use
% the .bib file (from which the .bbl file % is produced).
% REMEMBER HOWEVER: After having produced the .bbl file,
% and prior to final submission, you *NEED* to 'insert'
% your .bbl file into your source .tex file so as to provide
% ONE 'self-contained' source file.
%
% ================= IF YOU HAVE QUESTIONS =======================
% Questions regarding the SIGS styles, SIGS policies and
% procedures, Conferences etc. should be sent to
% Adrienne Griscti (griscti@acm.org)
%
% Technical questions _only_ to
% Gerald Murray (murray@hq.acm.org)
% ===============================================================
%
% For tracking purposes - this is V2.0 - May 2012

\documentclass{sig-alternate-05-2015}

\hyphenation{a-tu-al com-pe-ti-ti-vi-da-de di-fe-ren-ci-ais res-pos-tas co-nhe-ci-men-to i-de-a-li-za-do-res ca-mi-nho a-tra-vés ne-gó-cio 
re-fe-rên-cias ques-ti-o-ná-rios u-ma re-si-den-tes coo-pe-ra-ção
ca-rac-te-ri-za-ção e-co-no-mia con-si-de-rar }
\begin{document}

% Copyright
\setcopyright{acmcopyright}
%\setcopyright{acmlicensed}
%\setcopyright{rightsretained}
%\setcopyright{usgov}
%\setcopyright{usgovmixed}
%\setcopyright{cagov}
%\setcopyright{cagovmixed}


% DOI
%\doi{10.475/123_4}

% ISBN
%\isbn{123-4567-24-567/08/06}

%Conference
\conferenceinfo{SBSI 2017}{June 5$^{th}$ -- 8$^{th}$, 2017, Lavras, Minas Gerais, Brazil}

%\acmPrice{\$15.00}

%
% --- Author Metadata here ---
%\conferenceinfo{WOODSTOCK}{'97 El Paso, Texas USA}
%\CopyrightYear{2007} % Allows default copyright year (20XX) to be over-ridden - IF NEED BE.
%\crdata{0-12345-67-8/90/01}  % Allows default copyright data (0-89791-88-6/97/05) to be over-ridden - IF NEED BE.
% --- End of Author Metadata ---

\title{Impacto das Redes de Negócios para Startups: \\Um Estudo Empírico na IETEC/CEFET-RJ}
\subtitle{Alternative Title: The Impact of Business Networks for Startups: An Empirical Study at IETEC/CEFET-RJ}
%
% You need the command \numberofauthors to handle the 'placement
% and alignment' of the authors beneath the title.
%
% For aesthetic reasons, we recommend 'three authors at a time'
% i.e. three 'name/affiliation blocks' be placed beneath the title.
%
% NOTE: You are NOT restricted in how many 'rows' of
% "name/affiliations" may appear. We just ask that you restrict
% the number of 'columns' to three.
%
% Because of the available 'opening page real-estate'
% we ask you to refrain from putting more than six authors
% (two rows with three columns) beneath the article title.
% More than six makes the first-page appear very cluttered indeed.
%
% Use the \alignauthor commands to handle the names
% and affiliations for an 'aesthetic maximum' of six authors.
% Add names, affiliations, addresses for
% the seventh etc. author(s) as the argument for the
% \additionalauthors command.
% These 'additional authors' will be output/set for you
% without further effort on your part as the last section in
% the body of your article BEFORE References or any Appendices.

\numberofauthors{3} %  in this sample file, there are a *total*
% of EIGHT authors. SIX appear on the 'first-page' (for formatting
% reasons) and the remaining two appear in the \additionalauthors section.
%
\author{
% You can go ahead and credit any number of authors here,
% e.g. one 'row of three' or two rows (consisting of one row of three
% and a second row of one, two or three).
%
% The command \alignauthor (no curly braces needed) should
% precede each author name, affiliation/snail-mail address and
% e-mail address. Additionally, tag each line of
% affiliation/address with \affaddr, and tag the
% e-mail address with \email.
%
% 1st. author
\alignauthor
Rafael E. L. Escalfoni\\
       \affaddr{CEFET-RJ Nova Friburgo}\\
       \affaddr{Av. Gov. R. Silveira, 1900}\\
       \affaddr{Nova Friburgo, RJ, Brasil}\\
       \email{{\large rafael.escalfoni@cefet-rj.br}}
% 2nd. author
\alignauthor
Marcelo A. S. Irineu\\
       \affaddr{CEFET-RJ Maracanã}\\
       \affaddr{Av. Maracanã, 151}\\
       \affaddr{Rio de Janeiro, RJ, Brasil}\\
       \email{{\large marcelo.irineu@cefet-rj.br}}
% 3rd. author
\alignauthor 
Jonice Oliveira\\
       \affaddr{PPGI - UFRJ}\\
       \affaddr{CCMN, Cidade Universitária}\\
       \affaddr{Rio de Janeiro, RJ, Brasil}\\
       \email{{\large jonice@dcc.ufrj.br}}
%\and  % use '\and' if you need 'another row' of author names
% 4th. author
%\alignauthor Lawrence P. Leipuner\\
%       \affaddr{Brookhaven Laboratories}\\
%      \affaddr{Brookhaven National Lab}\\
%     \affaddr{P.O. Box 5000}\\
%       \email{lleipuner@researchlabs.org}
% 5th. author
%\alignauthor Sean Fogarty\\
%       \affaddr{NASA Ames Research Center}\\
%       \affaddr{Moffett Field}\\
%       \affaddr{California 94035}\\
%       \email{fogartys@amesres.org}
% 6th. author
%\alignauthor Charles Palmer\\
%       \affaddr{Palmer Research Laboratories}\\
%       \affaddr{8600 Datapoint Drive}\\
%       \affaddr{San Antonio, Texas 78229}\\
%       \email{cpalmer@prl.com}
}
% There's nothing stopping you putting the seventh, eighth, etc.
% author on the opening page (as the 'third row') but we ask,
% for aesthetic reasons that you place these 'additional authors'
% in the \additional authors block, viz.
%\additionalauthors{Additional authors: John Smith (The Th{\o}rv{\"a}ld Group,
%email: {\texttt{jsmith@affiliation.org}}) and Julius P.~Kumquat
%(The Kumquat Consortium, email: {\texttt{jpkumquat@consortium.net}}).}
\date{30 July 1999}
% Just remember to make sure that the TOTAL number of authors
% is the number that will appear on the first page PLUS the
% number that will appear in the \additionalauthors section.

\maketitle

\begin{resumo}
As \textit{startups} têm ganhado papel de destaque no cenário atual devido a seu desempenho para o lançamento de inovações com baixo custo e risco. Contudo, para assegurar o sucesso desses empreendimentos é necessário estabelecer um ambiente colaborativo para incentivar e dar suporte às atividades. Compreender e sistematizar tais requisitos do ecossistema de \textit{startups} é fundamental para estabelecer maior capacidade competitiva. O presente artigo apresenta um estudo empírico dos aspectos das diferentes estruturas de colaboração na incubadora IETEC CEFET-RJ. 
\end{resumo}

%%%%%%
% Use esse comando apenas se o artigo for em Português
%%%%%%
\palavraschave{Ecossistemas de \textit{Startups}; redes de negócios; inovação.}

\begin{abstract}
Startups companies have prominent role in the current business scenario. This is because they can get innovation with reduced costs and risks. However, to ensure the success of these ventures it is necessary to establish a collaborative environment to encourage and support those activities. Understanding and systematizing such startup ecosystem requirements is critical to establishing greater competitive capacity. This paper presents an empirical study of different structures of the collaboration aspects in incubator IETEC CEFET-RJ.
\end{abstract}


%
% The code below should be generated by the tool at
% http://dl.acm.org/ccs.cfm
% Please copy and paste the code instead of the example below. 
%
\begin{CCSXML}
<ccs2012>
	<concept>
		<concept_id>10002951.10003227.10003241.10003243</concept_id>
		<concept_desc>Information systems~Expert systems</concept_desc>
		<concept_significance>500</concept_significance>
	</concept>
	<concept>
		<concept_id>10002951.10003260.10003282.10003292</concept_id>
		<concept_desc>Information systems~Social networks</concept_desc>
		<concept_significance>300</concept_significance>
	</concept>
</ccs2012>  
\end{CCSXML}

\ccsdesc[500]{Information systems~Expert systems}
\ccsdesc[300]{Information systems~Social networks}


%
% End generated code
%

%
%  Use this command to print the description
%
\printccsdesc

% We no longer use \terms command
%\terms{Theory}

\keywords{Startup Ecosystem; business networking;  innovation}

\section{Introdução}
\textit{Startups} são empreendimentos associados à inovação e tecnologia que vêm ganhando papel de destaque no cenário atual em função da flexibilidade do modelo de negócios, baixos investimentos iniciais e maior dinamicidade nos ciclos de transferência tecnológica \cite{ries-2011}. Esta estrutura de negócios exerce  um papel fundamental no lançamento de novos produtos em um ambiente no qual a complexidade tem aumentado consideravelmente e despertado a necessidade de buscar novas estratégias para geração de inovações e diferenciais competitivos \cite{hardwick-et-al-2013}.

O estabelecimento de parcerias é essencial para a geração de inovações, pois o sucesso depende de um conjunto de infraestruturas e capacidades que abarcam desde a concepção do produto até a comercialização e distribuição no mercado. Além disso, a colaboração entre organizações é uma importante forma de lidar com as incertezas e riscos inerentes ao processo de inovação \cite{chesbrough-appleyard-2007, west-bogers-2014}. Frequentemente, empresas consolidadas não controlam esses ativos complementares, e, no caso das \textit{startups}, tal necessidade fica mais latente \cite{candido-souza-15}. 

As \textit{startups} têm maiores chances de sucesso quando se estabelecem em ecossistemas empreendedores, que estimulem o desenvolvimento de negócios e inovações \cite{spigel-2015}. Os ecossistemas de \textit{startups} são núcleos tecnológicos, onde parceiros compartilham recursos e habilidades específicas, através de apoio mútuo para estabelecer maior capacidade de competitividade em um cenário maior \cite{cukier-et-al-2015}. Desta forma, entender como tais relações de parcerias entre atores do ecossistema é crítico tanto da empresa quanto do próprio ambiente. O presente artigo apresenta um estudo feito na incubadora tecnológica do CEFET-RJ, a IETEC, onde foram identificados aspectos de colaboração entre os diferentes atores participantes deste ambiente de negócios, destacando os principais recursos compartilhados e auxiliando na identificação das características dos relacionamentos entre empreendedores. 

O restante deste artigo está organizado da seguinte forma: a Seção 2 apresenta os conceitos que fundamentam este trabalho. A Seção 3 detalha a caracterização da pesquisa - a elaboração do questionário utilizado, da IETEC e perfil dos entrevistados. A Seção 4 traz uma discussão sobre as respostas obtidas. Na Seção 5 são apresentadas as conclusões da pesquisa e os trabalhos futuros.\\


\section{Fundamentação Teórica}
\subsection{{\subsecit Startups}}
O termo \textquotedblleft startup\textquotedblright\ é utilizado para designar uma instituição temporária formada com o intuito de validar conceitos até que se tenha as condições mínimas para operacionalizar uma empresa, um projeto de organização que visa identificar um modelo de negócio consolidado e que possa ser expandido \cite{blankdorf-12}. Caso não haja sucesso em encontrar este caminho, a natureza do empreendimento permite interrompê-lo sem maiores prejuízos aos colaboradores e demais envolvidos, o que permite enfrentar maiores riscos e incertezas \cite{ries-2011}.

As novas ferramentas e metodologias de desenvolvimento, como o movimento \textit{Lean Startup}, propõem uma metodologia de experimentação e aprendizagem interativa. Seu ciclo de vida envolve um conjunto de passos que engloba desde o desenvolvimento de um conceito até o estabelecimento do negócio. Segundo Santos \cite{santos-2015}, as ações podem ser agrupadas nas seguintes fases: 

\begin{itemize}
	\item \textbf{Descoberta}: agrupa as ações para validar as hipóteses formuladas e para verificar se a solução ataca um problema relevante que consiga atrair o interesse das pessoas; 
	\item \textbf{Validação}: uma vez que o problema fica bem definido na etapa anterior, busca-se conhecer o segmento de clientes dispostos a pagar pelo produto. São feitos os ajustes necessários para tornar o produto rentável. Por vezes, recorre-se a ajustes para atender melhor às necessidades, estratégia conhecida como pivotagem.
	\item \textbf{Criação do Cliente}: até este ponto, o produto já foi desenvolvido e seu público-alvo já está bem definido. A criação do cliente está relacionada com a massificação do produto e a fidelização, o esforço para popularizar a marca. 
	\item \textbf{Construção da Empresa}: este estágio é a consolidação do empreendimento, quando a \textit{startup} se torna um negócio. Há uma estruturação da organização, com a definição dos papéis e responsabilidades.
\end{itemize}

Este tipo de empreendimento, tem sido tratado com grande expectativa por conta de seu potencial para o lançamento de inovações disruptivas \cite{weiblen-chesbrough-2016}. Devido a sua dinâmica, os custos de lançamento de produtos tornaram-se muito menores, trazendo à tona outros fatores de maior importância como por exemplo a inventividade e conhecimento de seus idealizadores e maior tolerância a riscos. 

Como forma de viabilizar tais negócios, um número crescente de instituições de apoio tem se estabelecido para orientar o negócio nos seus primeiros dias. São investidores e capitalistas de risco, incubadoras e aceleradoras, além de órgãos governamentais de suporte -- atores que compõem os ecossistemas de startups, provendo infraestrutura, espaço de mercado, parcerias consolidadas e processos estabelecidos~\cite{anthony-12}.

\subsection{Ecossistemas de {\subsecit Startups}}

Ecossistema de \textit{startups} são arranjos de atores (empreendedores, apoiadores, universidades, dentre outros) que buscam estabelecer os meios necessários para lançar novos produtos no mercado \cite{torres-souza-2016}. Essa comunidade funciona de maneira harmônica e dinâmica, tal qual um ecossistema biológico, onde o meio está constantemente se adaptando às mudanças, como  quando ocorre extinção de algum componente \cite{torres-souza-2016}. 

A rede de negócios é um dos aspectos principais para o cenário das \textit{startups}. O compartilhamento de ideias, experiências entre empreendedores novatos e experientes auxilia no aprendizado e desenvolvimento de competências essenciais para o sucesso de empresas. As conexões com organizações de suporte, como universidades e órgãos de fomento permitem acesso a recursos complementares para o desenvolvimento de produtos, tais como tecnologias e linhas de financiamento \cite{isenberg-2011} \cite{motoyama-waltins-2014} \cite{torres-souza-2016}. 

Para assegurar o equilíbrio e, consequentemente seu êxito em apoiar \textit{startups}, diferentes tipos de parcerias devem ser firmadas. Motoyama e Watkins \cite{motoyama-waltins-2014} estabelecem quatro conexões chaves para o ecossistema de \textit{startups}, são elas:

\begin{itemize}
	\item \textbf{Conexões entre empreendedores}: estabelecem uma comunidade de aprendizagem, onde as interações possibilitam a troca de saberes entre empreendedores com diferentes níveis de experiência. Os \textit{feedbacks} obtidos por meio deste tipo de rede proporcionam um  meio de aprendizado fundamental sobre o negócio.
	\item \textbf{Conexões entre organizações de suporte}: compreendem as relações estratégicas e funcionais entre instituições de pesquisa, universidades, incubadoras, dentre outras. Determinam o direcionamento dado ao ecossistema através da promoção de temas de interesse e no suporte de empreendimentos relacionados.
	\item \textbf{Conexões entre empreendedores e principais organizações de apoio}: Representam as parceiras firmadas entre \textit{startups} e as organizações de suporte. 
	\item \textbf{Demais conexões de apoio}: Evidenciam outros arranjos existentes entre empreendedores e atores diversos, como mentores, fornecedores e demais empresas parceiras.

\end{itemize}

\section{Abordagem Metodológica}
O presente estudo recorreu a evidências baseadas na experiência de especialistas com diferentes perspectivas de um ecossistema de \textit{startups}. Participaram através de questionários e entrevistas três gestores de empresas intensivas em conhecimento, um representante de um órgão de fomento e o coordenador de incubadora. 

No centro da aplicação do estudo encontra-se a IETEC -- a incubadora de empresas tecnológicas do CEFET-RJ. Fundada em 1994 com a proposta de apoiar empreendimentos na área de telecomunicação, desenvolvimento de \textit{hardware} e \textit{software}, a instituição passou por uma reformulação em 2010, passando a contemplar projetos tecnológicos e inovadores de todas as áreas de conhecimento mantidas pelo CEFET-RJ. Atualmente, a IETEC possui 8 projetos residentes, além de ter contribuído para a graduação de 18 empresas com um faturamento médio de R\$ 6,5 milhões. 

\subsection{Processo Investigativo}
%As entrevistas foram apoiadas por roteiros semiestruturados, constituídos de perguntas principais baseadas nas dimensões propostas por Motoyama e Watkins \cite{motoyama-waltins-2014} que evidenciam a relevância de diferentes tipos de redes de negócios dentro do ecossistema, sobretudo as conexões entre empreendedores e demais participantes do ambiente de negócios, dentro e fora da incubadora. Foram elaboradas questões abertas que buscavam identificar como as parcerias são formadas, como são prospectadas pelos gestores, se as startups contam com um processo de montar e manter suas parcerias, que tipo de recurso é intercambiado. Buscamos também medir o grau de influência nos negócios a partir de cada parceria, da relevância da incubadora, no mercado.

As entrevistas foram apoiadas por roteiros semiestruturados, constituídos de perguntas principais baseadas nas dimensões propostas por Motoyama e Watkins \cite{motoyama-waltins-2014} que evidenciam a relevância de diferentes conexões dentro do ecossistema, sobretudo as interações entre empreendedores e demais participantes do ambiente de negócios, dentro e fora da incubadora. 

Para isto, foram elaboradas questões abertas que buscavam caracterizar o ecossistema e identificar a relevância das relações: 
\begin{enumerate}
	\item \textbf{Conexões entre empreendedores: }(a) Quem são os principais parceiros? (b) O que leva a buscar cooperação? (c) Como são formadas e mantidas as parcerias? Há um processo formal de identificação de colaboradores? (d) Que tipo de recurso é compartilhado? (e) Que critérios são adotados para aceitação de parceiros? 
	
	\item \textbf{Conexões entre empreendedores e principais organizações de apoio: } (a) Quais as principais motivações para buscar apoio de instituições de suporte? (b) Quais são as organizações de fomento mais importantes para o seu negócio? (c) Como tais órgãos têm auxiliado o seu negócio? 
	
	\item \textbf{Demais conexões de apoio: } (a) Seu negócio mantém vínculos com outras organizações? (b) Qual é o papel destas parcerias? 
	
	\item \textbf{Conexões entre organizações de suporte:} (a) Possui parcerias com outros órgãos de apoio (quais)? (b) O que se busca neste tipo de parceria? (c) Como funcionam tais conexões?
\end{enumerate}

O tratamento de dados foi feito utilizando o método de Análise de Conteúdo, no qual buscou-se identificar padrões que possam surgir dos documentos analisados \cite{bardin-09}. O processo de análise foi conduzido em 3 etapas: 

\begin{itemize}
	\item \textbf{Pré-análise}: em um primeiro momento, o material das entrevistas e dos questionários foi organizado através de uma categorização. Para isto, as questões foram divididas nos seguintes assuntos principais: caracterização do agente participante, formação e prospecção de parcerias, percepção da importância de conexões formadas com empreendedores e instituições de apoio, procedimentos para a manutenção e estratégias de negócio voltadas para  as redes, papel e relevância das instituições apoiadoras para o negócio. As respostas obtidas foram separadas pelo perfil dos especialistas, levando em conta sua experiência empreendedora e o papel exercido na organização estudada.
	
	\item \textbf{Descrição analítica}: neste passo, buscou-se elucidar as opiniões a partir da construção de quadro de referências e da busca por sínteses coincidentes e ideias divergentes. Contrapomos as opiniões para identificar conflitos e então voltamos a procurar os participantes para tirar dúvidas; e 
	
	\item \textbf{Interpretação}: após tirar dúvidas, foram identificadas as conexão entre ideias e conhecimentos adquiridos ao longo do trabalho. A análise crítica destas relações e a base teórica que permeia este trabalho, formam a base das evidências empíricas deste estudo.
\end{itemize}

\section{Resultados Obtidos}
Conforme identificado pela pesquisa, as conexões entre empreendedores manifestam-se de diversas maneiras: através da trocas informais de conhecimento por meio do espaço de \textit{co-working}, por reuniões periódicas na incubadora e relações pontuais decorrentes de indicações. As interações, normalmente ocorrem para troca de experiências, para obter conhecimentos complementares ou para solucionar gargalos em processos operacionais. A literatura indica como uma das razões deste tipo de cooperação, a complementaridade da cadeia de valor \cite{candido-souza-15}. Conforme identificado, há relatos de parcerias firmadas com empresas consolidadas para este fim.

No ano passado, iniciou-se um programa de mentoria. Os projetos que participaram da dinâmica, passaram por um processo de compreensão dos propósitos de seus produtos e então foi feita a escolha do mentor mais adequado para cada caso. Empreendedores novatos passaram a se reunir com empresários experientes periodicamente por meio de videoconferências.  Este contato possibilitou melhorar a percepção do negócio, acarretando em reformulações de conceitos ou mesmo dando mais confiança ao conceito do produto. Alguns projetos montaram parcerias e conseguiram aportes financeiros e apoio por meio de consultorias. 

Com relação às conexões entre organizações de apoio, veri-ficou-se um ambiente fértil em ações para prover uma estrutura adequada para os empreendedores. A exemplo das manifestações, a IETEC está envolvida em consórcios entre incubadoras e parques tecnológicos e fóruns temáticos para promoção de inovação, onde tem participado ativamente de programas para intercâmbio de experiências. A instituição tem mantido parcerias com empresas, sobretudo as empresas graduadas, que atuam ativamente no programa de mentoria. 

O relacionamento entre empreendedores e organizações de apoio ocorre predominantemente por meio da incubadora. Porém, grande parte dos empreendedores mantém um vínculo sistemático com universidades e centros de pesquisa. O CEFET-RJ, por exemplo, possui um núcleo de inovação tecnológica, responsável por promover a inovação e transferência tecnológica, além de servir de interface da universidade com a incubadora. 

Os empreendedores têm conseguido aporte financeiro por meio de fomento. Para facilitar o acesso à informação, a IETEC tem mantido um mapeamento de editais e das políticas propostas dos órgãos de fomento. Este recurso tem sido usado para promover o alinhamento das \textit{startups} com as aspirações da Sociedade manifestadas pelas linhas de financiamento disponibilizadas.

O processo investigativo também almejou identificar as principais barreiras para a formação de parcerias. Verificou-se que a falta de planejamento é um problema recorrente, que dificulta a comunicação com os pares e a percepção de futuro para o negócio. O medo em expor as opiniões pode comprometer as relações: por vezes, o empreendedor tem receio de expor as ideias e ser plagiado - faltam mecanismos para distinguir as informações sensíveis para o negócio das que devem ser compartilhadas com os potenciais parceiros. Os problemas de comunicação dos interlocutores da \textit{startup} também foram relatados, assim como a falta de informações sobre potenciais parceiros, a falta de sinergia e problemas financeiros. 

\section{Conclusão e Trabalhos Futuros}
A inovação tem sido descrita como o único caminho para estabelecer um modelo de competividade sustentável na economia atual \cite{weiblen-chesbrough-2016}. Neste contexto, as \textit{startups} têm sido uma poderosa ferramenta de experimentação e aprendizagem, com um grande potencial para a geração de produtos com maior valor agregado. Contudo, o sucesso desse tipo de empreendimento é dependente do ecossistema no qual está inserido e das relações mantidas entre seus agentes \cite{blankdorf-12}\cite{candido-souza-15}\cite{motoyama-waltins-2014}. 

%Este artigo apresentou um estudo das ações promovidas dentro da IETEC/CEFET-RJ para criar e manter redes de parcerias entre diferentes atores do ecossistema de \textit{startups}. Conforme foi verificado, todas as conexões chaves mencionadas em Motoyama e Watkins \cite{motoyama-waltins-2014} estão sendo contempladas de alguma forma. Por outro lado, deve-se considerar o grau de sistematização das interações e a dependência dos agentes principais, sobretudo da incubadora. É preciso criar mecanismos para difundir melhor as informações sobre os recursos disponíveis na rede, a exemplo do que foi feito com o processo de acompanhamento e divulgação dos editais de fomento mencionado. Também não há um processo estruturado para a identificação e gestão da rede. Todas as atividades ocorrem de maneira \textit{ad hoc}, o que pode deixar grandes oportunidades dependentes da percepção dos envolvidos.

O presente artigo apresenta um estudo em andamento que busca compreender como ocorrem as interações entre atores de ecossistemas de \textit{startups} para aprimorá-las. Nesta primeira etapa da pesquisa, utilizou-se a IETEC/CEFET-RJ como objeto de estudo, onde foi feito um levantamento de como as relações ocorrem atualmente, vislumbrando oportunidades de aperfeiçoamento por meio de técnicas e ferramentas computacionais.

Os resultados obtidos através do estudo não podem ser generalizados, mas trazem indícios de que as conexões chaves mencionadas por Motoyama e Watkins \cite{motoyama-waltins-2014} estão sendo contempladas de alguma forma. Por outro lado, deve-se considerar o grau de sistematização das interações e a dependência dos agentes principais, sobretudo da incubadora. Não há um processo estruturado para a identificação e gestão da rede e todas as atividades ocorrem de maneira \textit{ad hoc}, o que pode deixar grandes oportunidades dependentes da percepção dos envolvidos. É preciso criar mecanismos para difundir melhor as informações sobre os recursos disponíveis na rede, a exemplo do que foi feito com o processo de acompanhamento e divulgação dos editais de fomento mencionado.

Em trabalhos futuros, esperamos ampliar nosso conhecimento acerca do universo das \textit{startups} por meio de uma análise investigativa mais detalhada. Estamos planejando um estudo envolvendo outras incubadoras, aceleradoras e outros parceiros da IETEC. Pretendemos explorar o planejamento de parcerias. O objetivo é auxiliar os participantes do ambiente empreendedor a terem uma visão mais ampla das possibilidades de parceria e promover a gestão de rede por meio de mecanismos de análise de redes sociais. Um exemplo de aplicação seria a identificação dos agentes principais da rede e os gargalos para cada participante. Também vislumbramos a utilização de sistemas de recomendação para identificar parceiros mais apropriados.


%ACKNOWLEDGMENTS are optional
\section{Agradecimentos}
Gostaríamos de agradecer à Incubadora de Empresas Tecnológicas IETEC CEFET-RJ por todo apoio ao projeto. Ao CNPq, CAPES e FAPERJ.

%
% The following two commands are all you need in the
% initial runs of your .tex file to
% produce the bibliography for the citations in your paper.
\bibliographystyle{abbrv}
\bibliography{eisi2017}  % sigproc.bib is the name of the Bibliography in this case
% You must have a proper ".bib" file
%  and remember to run:
% latex bibtex latex latex
% to resolve all references
%
% ACM needs 'a single self-contained file'!
%
%APPENDICES are optional
%\balancecolumns


\end{document}
