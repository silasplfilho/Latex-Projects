\chapter{Revisão da Literatura}\label{cap:rev_literatura}

For the literature selection we have applied a systematic literature review in order to have a deeper insight from the most actual research which tackles the depression diagnosis in social media. 

From a computational perspective, the problem of dealing with depression and social media can be understand as search for people who suffer the symptoms of depression. Using different methods to identify if someone have depression or not, or if some person have one of the symptoms of depression. A good amount of articles relies on natural language processing (NLP) to make a systemic analysis over the text in social media publications.

In \cite{Zhao:2018:TCM:3302425.3302501}, the authors have applied text classification technique using Convolutional Neural Networks to classify depression using text analysis. The authors in \cite{Nobles:2018:IIS:3173574.3173987} also have used neural networks in order to find patterns on periods when the risk of suicide attempt is increasing in SMS texts. 

The work in \cite{Li2016} is a qualitative study that try to understand how is the behavior and reception of chinese population about depression. It is a qualitative study and differs from the prior ones.

De Choudhury is an author that we would like to highlight. She is one author who have developed many articles and researches about measurement of depression on social media context. Her works vary on some types of measurement and implications of depression in some contexts of affected persons. We will cite some of these works due to the importance and relevance of her effort to contribute to this area. In \cite{DeChoudhury:2013:SMM:2464464.2464480}, the authors start to analyze might afflicted people by depression. They have used crowdsourcing to obtain data from twitter by people who were clinically diagnosed with depression. With this data, they have constructed a corpus and created a probabilistic model. The trained model would be relevant to indicate if a not seen post indicates depression.
Similar to previous work, Tsugawa et al \cite{Tsugawa2015} have applied the same analysis. However the users were selected from Japan. They try to replicate the results from \cite{DeChoudhury:2013:SMM:2464464.2464480}.

In \cite{Park:2015:MDL:2675133.2675139}, they present how activities on Facebook are associated with the depressive states of users and how depressive moods.  Their goal was to raise awareness to depression at the university where the study was conducted.

On \cite{andalibi_sensitive_2017}, the authors explore self-disclosures posts in Instagram. In this article, the authors use posts from people who tagged their post with \#depression in order to understand what kinds of sensitive disclosures do people make on Instagram.

In \cite{Homan:2014:SSD:2531602.2531704}, %\textbf{Social Structure and Depression in TrevorSpace}
the authors bring an approach to understand what is the behaviour of users in TrevorSpace. TrevorSpace is a social media platform (social network site - SNS) where their users are people from LGBTQ. This platform aims to prevent and avoid suicide among these users community.
% Even though suicide and depression can occur inside different groups and ages, the authors confirm that some groups can face this problem more frequently. The LGBTQ is assured to be one of these groups who face it.
% The authors choice on that SNS is justified by completeness of the network. Usually the task of selecting a certain group from a general network or social media 'esbarra' on selecting properly the users related to analysis. Thus, 'split' not related users to the desired ones is a reflection needed task. In this paper the authors don't have this problem because the whole network and its users are from the studied analysis group.

The author in \cite{Vedula2017} conduct an observational study to understand the interactions between clinically depressed users and their ego-network when contrasted with a group of users without depression. They identify relevant linguistic and emotional signals from social media exchanges to detect symptomatic cues of depression.

In \cite{Yazdavar:2017:SAM:3110025.3123028}, authors incorporates temporal analysis of user-generated content on social media for capturing symptoms. They developed a statistical model which emulates traditional observational cohort studies conducted through online questionnaires by extracting and categorizing different symptoms of depression by modeling user-generated content in social media.

In \cite{Chen2018}, they detected eight basic emotions and calculated the overall intensity (strength score) of the emotions extracted from all past tweets of each user. After that, created a time series for each emotion of every user to generate a selection of descriptive statistics for these time series. 
