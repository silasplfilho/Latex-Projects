\chapter{Introduction}\label{cap:introduction}

% A criação de soluções inovadoras é uma das ações esperadas de uma abordagem científica. De forma geral e simplista, é através da observação de fenômenos, naturais ou artificiais, que o homem tenta entender um problema. Através de hipóteses, testes, validações e experimentações, o chamado cientista tenta apresentar soluções para os mais diversos problemas.
The creation of innovation solutions is one of the expectations from what is called scientific approach. In a simple and general way, it is from observation of phenomenas, natural or artifical one, that someone, normally called scientist try to discover new solutions for different problems and challenges.

% Em alguns momentos, os problemas abordados pelos cientistas são facilmente aplicáveis e a sociedade adjacente ao cientista consegue enxergar uma aplicação para o esforço científico. Deveras, é fato que o ser humano busca melhorar questões relacionadas à sua saúde mental ou física. Seja para estender a ``vida útil'' de uma pessoa com muita idade, ou então trazer bem estar para aquele que sofre com alguma patologia.
Ocasionally, tackled problems by researches are easilly appliable and the society who witness can visualize possible applications of that effort. Indeed, there is a wider effort which tries to improve solutions, and solve new problems related to mental and physical health. Since for extend the quality of life, for someone who suffers from any pathology.

Fato é que a saúde mental das pessoas tem sido discutida nos mais diversos círculos sociais e em diferentes mídias. Infelizmente, mais reportagens tem demonstrado que diferentes círculos sociais e econômicos são afetados por uma falta de tratamento e cuidado da saúde mental. A Classificação Estatística Internacional de Doenças e Problemas Relacionados com a Saúde, que está em sua décima versão (CID 10), padroniza os mais diversos tipos de doenças e patologias, de modo a que tal padronização ajude profissionais a identificarem os elementos que caracterizem determinada doença. Ele também classifica as diversas patologias em categorias, de modo hierárquico, agrupando assim doenças correlatas e similares.
The fact is that

% Depressão é uma das doenças mentais que mais recorrentes no mundo. Alguns dirão que é depressão é o mal do século. Em situações extremas, essa patologia pode levar a pessoa a ideação suicida e até mesmo ao ato em si \cite{AmericanPsychiatryAssociationApa2013}. A Organização Mundial da Saúde (OMS) apresenta o que cerca de 300 milhões de pessoas no mundo, de diferentes idades possuem algum nível de depressão\footnote{www.who.int/en/news-room/fact-sheets/detail/mental-disorders}. No Brasil, o Ministério da Saúde estima que 11,5 milhões de brasileiros são afetados pela depressão\footnote{www.blog.saude.gov.br/index.php/materias-especiais/52516-mais-de-onze-milhoes-de-brasileiros-tem-depressao}.
Depression is one of the most related mental diseases in the world. Some people call it the century illness due to its dangerousness. It can lead, in extreme situations, to suicide\cite{AmericanPsychiatryAssociationApa2013}. World Health Organization (WHO) presents a number of around 300 million people from different ages who suffer some kind of depression\footnote{www.who.int/en/news-room/fact-sheets/detail/mental-disorders}. Some of these symptoms are, for example, depressed mood in most of the day, lost of interest in regular activities, weight loss, and insomnia. In Brazil, the Health Ministry presents a number of 11,5 million people who are affected by depression \footnote{www.blog.saude.gov.br/index.php/materias-especiais/52516-mais-de-onze-milhoes-de-brasileiros-tem-depressao}. 

Em alguns momentos, depressão é erroneamente entendida como unicamente um estado emocional de tristeza. De fato alguém diagnosticado como depressivo tem o humor alterado. No entanto, a duração de tal estado do humor da pessoa é que diferirá alguém corretamente diagnosticado com a patologia depressão, de alguém que devido a algum acontecimento, está temporariamente triste ou cabisbaixa.
% Sometimes depression is wrongly understood as a emotional state which reflects sadness. Independently of above examples, the fact is that the mental disease depression is wrongly defined and acknowledged.
%Depression is one of the most related mental disease in the world and some people call it as the century illness. World Health Organization (WHO) presents a number of around 300 million people from different ages who suffer some kind of depression \footnote{www.who.int/en/news-room/fact-sheets/detail/mental-disorders}. 

%Depression is classified on 11th international Disease Classification (ICD 11) as a disease when it is diagnosed in someone behavior. Some of these symptoms are for example, depressed mood in most part of the day, lost of interest in regular activities, weight loss and insomnia. 
% Depression can happen triggered by an event where the person has lost something e.g. job, some close person etc.
%This disease is also dangerous in order of its extreme consequences. Depression, according to ICD 11, can lead to suicide ideation and consequently to suicide \cite{AmericanPsychiatryAssociationApa2013}.

%With the wide spread and with more people with depression, it becomes a challenge identify a person who can be in a real state of depression. Depression also afflicts people who are in specific location where the access to a professional is \textbf{costly impending}. Thus, identify and attend someone who could be a potential depressive patient, in a fast and unobtrusive way, seems very helpful. 

Segundo a OMS, a tendência é que cada vez mais pessoas sejam afetadas pela depressão. Por conta disso, é um desafio atender a um crescente número de pessoas que de fato sofrem tal doença, ou então que minimamente, possuem tendência a serem depressivas. Podemos listar desafios como, a formação de profissionais que atendam diretamente essas pessoas, ou então meios de diagnóstico prévio, meios de estreitar a comunicação entre profissional e cliente. Portanto, seria de grande ajuda, identificar e atender potenciais depressivos de modo rápido e não intrusivo.
% With more affected people, it becomes a challenge to identify a person who can be in a real state of major depression disease. It also afflicts people who are in a specific location where access to a professional is hindered due to cost. Thus, it is helpful to recognize and attend someone who could be a potential depressive patient, in a fast and unobtrusive way.

Sobre métodos de diagnóstico online, \cite{Horvitz} cita os termos \textit{infodemiologia} e \emph{detecção digital de doenças} como métodos correlatos que usam plataformas digitais e ferramentas de tecnologia para melhorar a saúde da sociedade. Esses termos podem ser entendidos como esforços para atacar a detecção de epidemias, indivíduos que estão em risco e também comunicar possíveis afetados por alguma doença. O uso de tecnologia poderia diretamente apoiar instituições, profissionais e também a própria população de forma geral. Fazendo assim com que as pessoas tenham conhecimentos sobre uma doença, e assim tornando-as mais conscientes sobre causas e sintomas de alguma doença.
% In the context of online diagnosis, \textit{infodemiology} and \textit{digital disease detection} are correlated terms to describe the use of digital platforms and tools to improve society health. They can be translated as efforts to tackle epidemics detection, identify individuals at risk, and communicate candidate urgent illness. The use of technology could directly support institutions, professionals, and even to aid people to make themselves aware of some disease\cite{Horvitz}.%, in order to allow aware inside society. In that way, people could be aware about some disease and its implications.
Sobre como identificar alguém com depressão, a Psicologia já possui métodos de diagnóstico e comumente é feito através de entrevistas do psicólogo com o paciente/cliente. Através dessa entrevista, o psicólogo identifica características no paciente que correlacionam com os sintomas da doença.
% Psychology already has methods to deal with depression identification. One of these methods is interview with psychologist. Through this conversation it could be possible to the professional, find some clues to identify someone with depression symptoms. 

% The abstraction of that is the question, how someone express himself? Some people might express through text, others will prefer with music, others will prefer to not externalize some feelings and sentiments about himself. This makes the task of identification more challenge.
% Nowadays, some groups of people have demonstrated an interesting behaviour. The act of express itself has been demonstrated on social media. People have express themselves more on these online platforms.  

Social media has been employed on academia to monitor people's behaviour and their personal choices. With this in mind, sounds interesting to investigate if it is possible to identify signs, symptoms of depressive behaviour on social media platforms.

As said above, the task of identifying some disease, even if it is not depression, can be challenging. Due to the plant offers of data types, select what is the most effective, precise and reliable technique, method analysis could require a great quantity of time of research.
 
Thus, we can abbreviate our effort as the following question: \emph{Is it possible to identify psychological diseases symptoms from people who are social media users?}
% . Recorte de interacoes de pessoas com sintomas de depressao em mídias sociais
% . Grupo de trabalhos sobre métricas e metodos de análise
Our research questions can be described as follow
\textit{It would be possible to identify psychological diseases symptoms using social media?}
% . Seria possivel identificar sintomas de doencas psicologicas por meio das midias sociais?	 
Is there a disturb diagnosis method only using data from social media? 


%As interações entre empresas distintas e outras entidades que colaboram entre si para compartilhar tecnologias, conhecimentos e trabalhar em conjunto são fundamentais para o desenvolvimento de novos produtos e serviços, fato corroborado por Ben Spigel \cite{spigel-2015}, que afirma que a inovação depende de um ambiente favorável para se desenvolver e potencializar a geração de outras mais, atraindo pessoas, recursos e oportunidades de fomento. 

%Esta rede de negócios tem sido analisada sob a ótica de ecossistemas, conforme inicialmente proposto por James F. Moore \cite{moore-1993}. A analogia estabelecida pelo autor procura estabelecer pontes das relações de parcerias entre os diferentes atores e o meio onde se inserem, considerando as mudanças constantes, como o surgimento de novos participantes, tecnologias e mercados e extinção de outros. Quando ocorrem de maneira integrada e dinâmica, possibilitam a evolução do ecossistema como um todo. 
%Em um ecossistema de startups, a comunidade é formada por empreendedores, apoiadores, universidades e outros grupos de interesse, que realizam atividades econômicas e negócios para estabelecer os meios necessários para lançar novos produtos no mercado \cite{lemos:2011, torres-souza:2016}. 

%As conexões em um ecossistema de startups são pautadas por dois tipos de interação: a colaboração e a competição. A colaboração é uma importante maneira de lidar com as incertezas inerentes do processo de inovação, através do compartilhamento dos riscos  \cite{chesbrough-appleyard:2007, west-bogers-2014}. 

%Além do mais, o sucesso de empreendimentos depende de um conjunto de infraestruturas e capacidades que abarcam desde a concepção do produto até a comercialização e distribuição no mercado. Frequentemente, as empresas consolidadas não controlam estes ativos complementares, no caso das startups, tal necessidade fica mais latente \cite{candido-souza-15}. 
%As relações de competição decorrem do princípio de que duas ou mais empresas possuem interesses divergentes onde uma empresa apenas obterá vantagem competitiva às custas de seus concorrentes. Este tipo de conexão, onde empresas cooperam em determinadas atividades e competem em outras é denominada \textquotedblleft coopetição\textquotedblright. A coopetição possui grande relevância para a geração de inovações e vêm motivando uma série de estudos que buscam compreender como pode ser gerenciado \cite{gubbins-dooley:2014, chesbrough-appleyard:2007, bengtsson-kock:2000}. 

%\textit{\textbf{outro importante fator ao abordar ecossistemas de startups é a coopetição - um modelo[?] onde empresas se organizam em estruturas de colaboração e por vezes competem entre si. }}

%O termo \textquotedblleft rede social online\textquotedblright\ é utilizado para descrever um grupo de pessoas que interage primariamente através de alguma mídia de comunicação. As redes sociais online têm facilitado a colaboração, o compartilhamento e a interação e tem aumentando a quantidade de informação de tal forma que existe muito mais informação sobre qualquer assunto do que uma pessoa seria capaz de absorver \cite{castells-96}.  Bevenuto e seus colegas \cite{bevenuto-2011} tratam rede social online como um serviço web que permite a construção de perfis públicos ou semi-públicos dentro de um sistema, a articulação de ações entre usuários que compartilham conexões.

%, tais como o grau de vértices, coeficiente de agrupamento, distância média, centralidade de nodos e reciprocidade,

\section{Problema}

%Para que funcionem de forma efetiva, é necessário entender como as relações são orquestradas. Neste sentido, uma série de paradigmas têm sido propostos para ampliar a compreensão das interações entre os diferentes atores das redes de inovação, tais como: o modelo da hélice tripla proposto por Etzkowitz \cite{etzkowitz:2008}, a estratégia de cocriação idealizada por Phahalad e Ramaswamy \cite{prahalad-ramaswamy-2004}, o modelo de ecossistemas de negócios, analogia criada por Moore \cite{moore-1993}.


%Conforme mencionado na Seção~\ref{sec:motivacao},

\section{Hipótese}
%A hipótese deste trabalho consiste na possibilidade de definição de um conjunto de técnicas para sistematizar o processo de análise e recomendação de parcerias em um ecossistema de startups através da Teoria de Grafos \cite{newman:2010}. A análise da rede visa a compreensão da dinâmica em que são firmadas ou extintas as conexões. Tal esforço possibilita a identificação de ações que possam aprimorar as relações, buscando a convergência de interesses para aumentar a resiliência do ambiente \cite{spigel-2015}. Para isto, consideram-se que os atores e as relações de parcerias existentes podem ser mapeadas como uma rede social heterogênea, onde diferentes atores possuem objetivos específicos no arranjo.

% A hipótese deste trabalho consiste na possibilidade de definição de um conjunto de métricas para auxiliar o processo de análise das relações de parcerias de um ecossistema de startups através da Teoria de Grafos \cite{newman:2010}. A análise da rede visa a compreensão da dinâmica em que são firmadas ou extintas as conexões, seus principais atores e possíveis ameaças. Tal esforço possibilita a identificação de ações que possam aprimorar as relações, buscando a convergência de interesses para aumentar a resiliência do ambiente \cite{spigel-2015}. Para isto, consideram-se que os atores e as relações de parcerias existentes podem ser mapeadas como uma rede social heterogênea, onde diferentes atores possuem objetivos específicos no arranjo.


%A hipótese deste trabalho se apresentará em duas partes. Na primeira, consideram-se que os atores e as relações de parceria existentes podem ser mapeadas como uma rede. A compreensão da dinâmica em que são firmadas ou extintas as conexões é fundamental para identificar ações que possam aprimorar as relações, buscando a convergência de interesses para aumentar a resiliência do ecossistema \cite{spigel-2015}. 

% Outra importante componente do estudo aborda a complementaridade necessária para cada participante do meio; \cite{west-bogers-2014} sugerem que há diferentes níveis de recursos externos compartilhados entre atores na busca por inovações. Cada participante da rede deve conhecer sua relevância frente a seus pares e que influências exerce \cite{easley-kleinberg:2010}. Desta forma, é preciso traçar estratégias de parceiras e identificar novos colaboradores que possam trazer novos recursos.  

% consiste na possibilidade de definição de um conjunto de técnicas que auxiliem na análise das parcerias existentes, recomendação de novas parcerias, identificação de conhecimentos desejáveis e as dependências na rede avaliação de relações de poder.

% Portanto, pretende-se utilizar as ferramentas de Análise de Redes Sociais para construir um modelo que, através de métricas estabelecidas, possa auxiliar nos processos de tomadas de decisões dos participantes do ecossistema de startup. Acreditamos que o mapeamento do cenário resultante possibilitará a identificação dos interesses envolvidos, os recursos disponíveis, as afinidades e aptidões da rede, possibilitando a recomendação de novos laços. 

%Em seguida, \textit{\textbf{[O uso de análise de redes sociais permitirá ao participante identificar que outros participantes podem trazer complementaridade para seus negócios, avaliar sua relevância na rede, seu poder de barganha e de influência, além de possibilitar traçar novos contatos para evitar dependências.]}}


\section{Objectives}
\label{sec:objectives}

Esta pesquisa tem ainda como objetivos específicos:

\section{Proposal Structure}
\label{sec:organization}
Este documento está estruturado da seguinte forma. O capítulo atual descreve a problemática abordada, e a hipótese que será abordada pela proposta de solução.
O Capítulo \ref{cap:depression} descreve a doença depressão sob o ponto de vista da psicologia, que é o campo de estudo apropriado.
O Capítulo \ref{cap:rev_literatura} apresenta o processo de mapeamento sistemático da literatura de forma a entender o atual estado da arte da computação para depressão.
No Capítulo \ref{cap:proposta}, é apresentada uma proposta de abordagem para identificar os sintomas da depressão numa mídia social. Sucessivamente, concluímos o documento no Capítulo \ref{cap:conclusao}.