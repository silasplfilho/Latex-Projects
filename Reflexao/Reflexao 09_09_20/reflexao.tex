\documentclass[12pt, legalpaper]{article}
\usepackage[utf8]{inputenc}

\begin{document}
Faz bastante tempo desde a última vez que estive numa situação parecida. Trazer a reflexão é desafiador e as vezes a vontade é fugir. Curioso é ver como o tempo permite analisar o que realizei em tal periodo.

Baseado nas ultimas reflexões e mensagens das semanas passadas, somos convidados a refletir sobre nossa posição como seguidores de Cristo. Especialmente nas 4as feiras, ouvimos sobre:

\begin{itemize}
   \item fazermos escolhas (usando o exemplo de Pedro ao tentar caminhar sobre as águas);
   \item a consequência das nossas escolhas e acoes (Ez 30 sobre o povo q não procede conforme aprende e conhece)
   \item a aliança criada por Deus com o seu povo, e a decisão do povo que é similar e exemplifica nossa atitude de ignorar muitas vezes a aliança
   \item a dificuldade do jovem cristão em enfrentar o pecado (culto jovem), sua auto aceitação, e o processo de construção da sua personalidade
   \item a crise de salomão a respeito da finitude das nossas acoes comparada com tudo aquilo que ocorrerá, e o desperdício do tempo em coisas terrenas.
\end{itemize}
 
Se analisarmos friamente, vemos que os temas são todos confrontadores e talvez, dependendo de quem escuta, até mesmo desanimadores. Não me excluo dessa frase, porque olhando para mim, vejo que me encaixo em várias, se não todas, as mensagens citadas. Recuso a aliança, minhas escolhas geralmente refletem a minha vontade carnal, isso quando as tomo e não fujo de decidir, me preocupo com meus próprios interesses, e não consigo muitas vezes aceitar e entender quem sou. 

Gosto bastante do cenário expositivo do pastor Craig Groeschel no seu livro confissões de um pastor. Pois mostra a fragilidade em alguém que para nós deveria ser a última pessoa a demonstrá-la. Não pela exposição em si, mas pela humanização na figura da pessoa descrita. O que pretendo nessa reflexão é tentar trazer alguma coisa que se assemelhe com a abordagem do livro. Para mim, a reflexão precisa conter algo que me impactou primeiramente quem a traz. Portanto, essa reflexão me fez relembrar certos pontos após ouvir as mensagens anteriores.

\newpage
Texto base: Hebreus 2:6-8

\emph{``Mas alguém em certo lugar testemunhou, dizendo: "Que é o homem, para que com ele te importes? E o filho do homem, para que com ele te preocupes? Tu o fizeste um pouco menor do que os anjos e o coroaste de glória e de honra; tudo sujeitaste debaixo dos seus pés". Ao lhe sujeitar todas as coisas, nada deixou que não lhe estivesse sujeito. Agora, porém, ainda não vemos que todas as coisas lhe estejam sujeitas."}
\\
Proponho nessa reflexão três movimentos. Um posicionamento do ser humano, um movimento descendente e o reestabelecimento.

\section{Primeiro Movimento - Estabelecimento}
\subsection{Pontos sobre o texto:}
 \begin{itemize}%[font=$\bullet$~\normalfont\scshape]
    \item O autor, então desconhecido, do texto em Hebreus afirma sobre como Deus permite ao homem dominar sobre diversos aspectos da natureza criada por Deus.
    \item É super interessante e cativante a capacidade do homem, ou como citado no texto “filho do homem”, como alusão ao ser humano em si. Somos capazes de aprender, dominar e criar. 
   \item Desde a antiguidade podemos ver construções como marcos do desenvolvimento humano. Impérios foram construídos. Sociedades foram consntruídas.
\end{itemize} 

O ser humano conseguiu conquistar muitas coisas ao longo da sua história. Na bíblia temos exemplos de personagens e contextos históricos como prova disso. 
O reino de Davi, Salomão, o império babilônico, o império romano. A herança de tais épocas refletem no presente através da matemática, astronomia, sociologia, filosofia e outros campos. Tudo isso como comprovação de que o homem é capaz de aprender, dominar e criar.

O homem, ou ser humano, como criação divina foi estabelecido para dominar sobre a criação. Vemos o salmista afirmar sobre tal posicionamento do ser humano.

\emph{pergunto: Que é o homem, para que com ele te importes? E o filho do homem, para que com ele te preocupes? Tu o fizeste um pouco menor do que os seres celestiais e o coroaste de glória e de honra. Tu o fizeste dominar sobre as obras das tuas mãos; sob os seus pés tudo puseste: Todos os rebanhos e manadas, e até os animais selvagens, as aves do céu, os peixes do mar e tudo o que percorre as veredas dos mares. Salmos 8:4-8}

\emph{Senhor, que é o homem para que te importes com ele, ou o filho do homem para que por ele te interesses? O homem é como um sopro; seus dias são como uma sombra passageira. Salmos 144:3,4}

\newpage
\section{Segundo Movimento - Descendente}
Apesar de todas as coisas estabelecidas pela humanidade que poderíamos listar. Ainda assim, se olharmos por determinados ângulos, começamos a descontruir aquilo que foi estabelecido. Se observarmos diferentes aspectos do ser humano indivíduo, ou sociedade, podemos também listar diversos aspectos que descontroem aquilo discutido anteriormente.

Diversos artistas já retrataram uma visão sobre o homem. Na obra ensaio sobre a cegueira de Saramago, temos a exposição do egocentrismo. Ao longo da narrativa temos casos violência, abuso, injustiça e abandono. 
O poço, watchmen, 1984, a revolução dos bichos entre outros tentam expor o mesmo. O quanto uma sociedade pode decair.
Saindo da ficção/realidade, observando a história, e a atualidade, também vemos fatos que desanimam. Quantas tragédias, injustiças, mortes ocorrem diariamente? Não teríamos tempo para listar.

A própria bíblia também mostra situações onde o leitor é exposto a coisas absurdas. 

\begin{itemize}
   \item Davi ao permitir a morte de Urias
   \item A depravação do povo de Sodoma
   \item Jonas ao ser impiedoso com Nínive
   \item Esquartejamento
\end{itemize}

\textbf{O homem olhando para si mesmo não consegue ter esperança para o futuro.}
Trago essa afirmação porque o homem ressalta mais os seus aspectos negativos. Quantas notícias boas ouvimos?

Não quero trazer também que devemos ser super positivos e ignorar a nossa realidade. Não falo em auto enganação. 

Alguns pela falta de esperança, adotam filosofias e estilos que tentam ignorar, ou abraçar totalmente a finidade humana e sua deterioração. O niilismo, filosofia, nega a existência de um motivo ou razão pela qual o homem existe. O hedonismo, por conta da finidade humana, prega que devemos fazer aquilo que nos dá prazer. 

\emph{Por que é que devo me preocupar com a política, economia e educação? Vai tudo continuar a mesma coisa.} 
O texto citado na mensagem do último domingo pela manhã mostra Salomão de certa forma desiludido. 

\textbf{O ser humano, ao se olhar no espelho, pode não gostar do que vê.}
E por causa disso, invento tarefas, atividades, pensamentos, finalidades, para o meu eu ficar supostamente satisfeito, e talvez achar um sentindo. Nem que seja mínimo.

\newpage
\section{Terceiro Movimento - Reestabelecimento}

No entanto, o centro da reflexão que proponho não se limita a olhar a realidade humana.
Na verdade, o que eu gostaria de trazer é a minha impressão sobre como a bíblia apresenta o olhar de Deus sobre o homem. O versículos seguintes do texto lido em Hebreus traz um aspecto do olhar divino. 

\emph{Vemos, todavia, aquele que por um pouco foi feito menor do que os anjos, Jesus, coroado de honra e glória por ter sofrido a morte, para que, pela graça de Deus, em favor de todos, experimentasse a morte. Ao levar muitos filhos à glória, convinha que Deus, por causa de quem e por meio de quem tudo existe, tornasse perfeito, mediante o sofrimento, o autor da salvação deles. Hebreus 2:9,10}

O texto traz a figura de Jesus como salvador. Trazido pelo Pai por meio da graça. Na verdade, se fosse resumir aquilo que gostaria refletir numa só palavra seria \textbf{graça}.

Em alguns trechos da bíblia me chamam a atenção ppor conta da graça de Deus sobre a minha vida.

Paulo mostra em diversas epístolas a condição do homem pecador, e nascido em pecado. Eu poderia falar tranquilamente letra por letra como fossem minhas as palavras.
\emph{Sei que nada de bom habita em mim, isto é, em minha carne. Porque tenho o desejo de fazer o que é bom, mas não consigo realizá-lo. Pois o que faço não é o bem que desejo, mas o mal que não quero fazer, esse eu continuo fazendo. Romanos 7:18,19}

No entanto a bíblia, e o evangelho não são instrumentos para punir o homem. Pelo contrário, Deus, na sua palavra, em todos os momentos me chama a reflexão sobre quem eu sou e entender minha situação como ser humano. Não através de falsas verdades, mas mostrando que mesmo que o homem tenha caído, Deus ainda assim derrama o seu amor sobre a minha vida e a sobre a vida de qualquer um que o buscar. Isso inclui toda e qualquer pessoa. Não sou eu quem designo.

Um dos momentos mais bonitos para mim, e que exemplifica o que tento dizer é quando Pedro é questionado por Jesus três vezes se o amava. O mesmo número de vezes que Pedro tinha negado Jesus.
Uma das coisas mais belas que o evangelho nos trás é a valorização do ser humano. Vemos Jesus ter compaixão da mulher adúltera. Compaixão dos que tinham fome. Compaixão do seu povo por rejeitá-lo. Todos esses casos comprovando a sua palavra em Jo 3:16:

\emph{"Porque Deus tanto amou o mundo que deu o seu Filho Unigênito, para que todo o que nele crer não pereça, mas tenha a vida eterna." João 3:16}

% e também sobre a grandiosidade do amor perante a fé e a esperança.

Quantas vezes já errei contra Deus. Já o desobedeci, já tentei fugir da sua presença. Ainda assim, Deus me perdoa muito mais que as 490 vezes citadas em Mateus. 

\emph{Então Pedro aproximou-se de Jesus e perguntou: "Senhor, quantas vezes deverei perdoar a meu irmão quando ele pecar contra mim? Até sete vezes? "Jesus respondeu: "Eu lhe digo: não até sete, mas até setenta vezes sete. Mateus 18:21,22}

\newpage

Ainda assim, Deus permite que eu me aproxime dEle. Deus permite que eu invoque o seu nome no mais profundo abismo da minha culpa. Como não me admirar, quando o seu evangelho permite não apenas salvar o homem do pecado, mas também permitir que ele consiga avançar no processo de autoconstrução? 

Me impressiono quando leio o texto do publicano e fariseu em Lucas 18. E em muitas vezes me identifiquei como fariseu ao falar todos os meus feitios. 

\emph{O fariseu, em pé, orava no íntimo: ‘Deus, eu te agradeço porque não sou como os outros homens: ladrões, corruptos, adúlteros; nem mesmo como este publicano. Jejuo duas vezes por semana e dou o dízimo de tudo quanto ganho’. Lucas 18:11,12}

Ainda assim, o próprio Deus, como homem, conversou com um outro fariseu, Nicodemos, e afirmou: ``necessário é nascer de novo".

\section{Conclusão}
Minha intenção com essa reflexão é tentar trazer aos irmãos uma motivação para o caminhar. Não como um discurso de autoajuda. Mas como apoio para tudo aquilo que temos ouvido e aprendido. Gosto muito da questão da graça porque ela é a ação de Deus que permite ao homem recomeçar. É através da graça, manifesta em Cristo que podemos tomar atitudes tais como nos capítulos seguintes de Hebreus, e já citada em Jeremias 31:

\emph{``Esta é a aliança que farei com eles, depois daqueles dias, diz o Senhor. Porei as minhas leis em seus corações e as escreverei em suas mentes"; e acrescenta: ``Dos seus pecados e iniqüidades não me lembrarei mais". Onde essas coisas foram perdoadas, não há mais necessidade de sacrifício pelo pecado. Portanto, irmãos, temos plena confiança para entrar no Santo dos Santos pelo sangue de Jesus, por um novo e vivo caminho que ele nos abriu por meio do véu, isto é, do seu corpo. Hebreus 10:16-20}

Esse trecho faz exalar a renovação por conta do perdão dos pecados. 

Creio que tal mensagem, a graça remissora e construtiva de Cristo, deva ser se não a principal, mas uma das principais mensagens a serem anunciadas. É ela quem impulsiona a santificação consciente. É ela que faz o homem ter compaixão do próximo, a compreender ou querer compreender aquele a quem Cristo não tocou ainda. Paul Tornier em seu livro Culpa e Graça afirma: 

\emph{``...todos nós somos, não alternadamente, mas ao mesmo tempo, acusados e acusadores, condenados e condenadores."}

Ao olhar para a minha ferida, me lembro do meu estado de doente. No entanto, a graça me faz lembrar que Cristo é o médico dos médicos. Aquele que vem tratar das minhas feridas. Feridos pelas ansiedades, dúvidas, injustiças entre outras coisas. Feridos pelo caos, pela injustiça causadas por mim mesmo, pelas minhas atitudes.
\end{document}