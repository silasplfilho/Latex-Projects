\documentclass[12pt, legalpaper]{article}
\usepackage[utf8]{inputenc}

\begin{document}

\textbf{Texto Base: João 4:5-26}

% \emph{Assim, chegou a uma cidade de Samaria, chamada Sicar, perto das terras que Jacó dera a seu filho José.}

% \emph{Havia ali o poço de Jacó. Jesus, cansado da viagem, sentou-se à beira do poço. Isto se deu por volta do meio-dia.}

% \emph{Nisso veio uma mulher samaritana tirar água. Disse-lhe Jesus: "Dê-me um pouco de água".}

% \emph{(Os seus discípulos tinham ido à cidade comprar comida.)}






% \emph{Disse a mulher: "Senhor, vejo que é profeta.}

% \emph{Nossos antepassados adoraram neste monte, mas vocês, judeus, dizem que Jerusalém é o lugar onde se deve adorar".}

% \emph{Jesus declarou: "Creia em mim, mulher: está próxima a hora em que vocês não adorarão o Pai nem neste monte, nem em Jerusalém.}

% \emph{Vocês, samaritanos, adoram o que não conhecem; nós adoramos o que conhecemos, pois a salvação vem dos judeus.}

% \emph{No entanto, está chegando a hora, e de fato já chegou, em que os verdadeiros adoradores adorarão o Pai em espírito e verdade. São estes os adoradores que o Pai procura.}

% \emph{Deus é espírito, e é necessário que os seus adoradores o adorem em espírito e em verdade".}

% \emph{Disse a mulher: "Eu sei que o Messias ( chamado Cristo ) está para vir. Quando ele vier, explicará tudo para nós".}

% \emph{Então Jesus declarou: "Eu sou o Messias! Eu, que estou falando com você".}

\begin{large}Introdução\end{large}

\begin{itemize}
   \item Como pequenas conversas, momentos curtos de diálogo, podem trazer revelações e orientações para nossa vida. Ontem a noite estava observando sobre como hábitos que possuo já eram feitos por alguém duas gerações antes de mim.
  
   \item O texto base traz um diálogo que talvez tenha sido até curto no momento em que foi ocorrido, mas com grande impacto atual e conteúdo riquíssimo para nossa vida.

   \item Um problema atemporal é o uso de máscaras na nossa vida.

\end{itemize}

\textbf{Trabalhar sobre como Deus consegue enxergar através das nossas máscaras, protocolos e muros.}
\begin{itemize}
   \item O texto pode nos levar a reflexão de temas distintos, sobre adoração, sobre o verdadeiro culto, sobre evangelismo, capacitação. Mas gostaria de tratar sobre nossa situação perante Deus do ponto de vista da mulher samaritana.
   
   \item Uma das características é que a palavra de Deus se aplica ainda hoje e muito atual.
   
   \item Escolhi esse texto por causa de como Jesus se aproxima da mulher samaritana. Sutilmente Jesus pede por água, indaga, questiona, expõe e soluciona.

   \item Somos apresentados ã figura da mulher samaritana, que no caso não possui nome. Não sabemos sua identidade, profissão, idade. Sabemos apenas qual província ela pertence e que vai buscar água no poço.

   \item Jesus, incialmente desconhecido pela mulher mas não por nós, pede por um refresco. E então somos apresentados aos primeiros problemas do cenário.   
  
   \emph{A mulher samaritana lhe perguntou: "Como o senhor, sendo judeu, pede a mim, uma samaritana, água para beber?" (Pois os judeus não se dão bem com os samaritanos.)}

   \emph{Jesus lhe respondeu: "Se você conhecesse o dom de Deus e quem lhe está pedindo água, você lhe teria pedido e ele lhe teria dado água viva".}

   \emph{Disse a mulher: "O senhor não tem com que tirar a água, e o poço é fundo. Onde pode conseguir essa água viva?}

   \emph{Acaso o senhor é maior do que o nosso pai Jacó, que nos deu o poço, do qual ele mesmo bebeu, bem como seus filhos e seu gado?"}


   \newpage
   \item Os impeditivos para a aproximação de Jesus à mulher samaritana eram:
   \begin{itemize}
      \item Barreira étnica (judeus vs samaritanos) - Racismo - Grupo que teve mistura etnica com outros povos
      \item Barreira de gênero (Jesus homem próximo de uma mulher) - Misoginia
      \item Barreira religiosa (onde é o local correto da adoração?) - Legalismo Religioso
      % \item O histórico da mulher samaritana com outros relacionamentos - talvez impeditivo dela para com as pessoas samaritanas - fruto de preconceito
   \end{itemize}

   \item Interessante é ver como Jesus a 2000 anos atrás já quebrava barreiras que ainda hoje temos.
   
   \item Jesus então apresenta para a mulher uma solução miraculosa para a necessidade constante por água:

   \emph{Jesus respondeu: "Quem beber desta água terá sede outra vez,}

   \emph{mas quem beber da água que eu lhe der nunca mais terá sede. Pelo contrário, a água que eu lhe der se tornará nele uma fonte de água a jorrar para a vida eterna".}

   \emph{A mulher lhe disse: "Senhor, dê-me dessa água, para que eu não tenha mais sede, nem precise voltar aqui para tirar água".}

   \item Não quero abordar o problema do ponto de vista acusativo, mas gostaria de convidar você a refletir tal como fui incomodado. Na verdade, todos nós estamos suscetíveis a utilizar máscaras. Como forma de estabelecer um posicionamento social, talvez como forma de autoproteção, como modo de influência, por tradição, medo de perder uma suposta segurança.
   
   \emph{Ele lhe disse: ``Vá, chame o seu marido e volte".}

   \emph{``Não tenho marido", respondeu ela. Disse-lhe Jesus: ``Você falou corretamente, dizendo que não tem marido.}

   \emph{O fato é que você já teve cinco; e o homem com quem agora vive não é seu marido. O que você acabou de dizer é verdade".}

   \item Vemos aqui uma exposição da mulher samaritana. Se estamos usando o exemplo da mulher samaritana, onde é que Jesus está expondo os meus erros? Muitos talvez ao saber que seriam expostos em algum cenário podem ficar tímidos, reprimidos. Imagine se agora eu comece a falar todos as minhas falhas. Seria isso relevante?

\newpage
   \item Talvez a exposição não ocorra aqui, mas ocorra quando vocês está sozinho, quando está com outra pessoa.
   
   \item Mas é importante destacar que a exposição não acontece para a humilhação, para a destruição ou detrimento da pessoa. Mas ela ocorre para o crescimento. Para a quebra de máscaras. Para 

   \item Assistimos hoje o documentário Dilema das Redes, disponível na netflix. No documentário é mostrado como as mídias sociais podem impulsionar um determinado comportamento virtual, de consumo, interação e informação. Não restringindo-se apenas ao ambiente virtual, mas trazendo também consequências no ambiente real, do dia a dia. Logo, fazendo com que criemos também máscaras virtuais. Criando uma imagem, um avatar, um perfil que pode ter desejos de quem gostaríamos ser, que se conecta com quem gostaríamos de nos envolver. 


   \begin{large}Conclusão\end{large}

   \item Uma das características do evangelho é a inclusão daqueles que de alguma forma são excluídos. 
   Os leprosos que são afastados do convívio, os gentios, os corruptos, os adúlteros. 

   \item Na verdade, Jesus veio para todo aquele que nele crer. Isso envolve os excluídos e os incluídos socialmente. Na verdade, vemos através da mulher samaritana que toda ser humano é carente e é alvo do amor de Deus.

   \item O

\end{itemize}


\end{document}