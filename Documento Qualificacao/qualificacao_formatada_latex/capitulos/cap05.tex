\chapter{Metodologia de Pesquisa}\label{cap5}

%A natureza da pesquisa sugere a necessidade de aproximar a teoria da prática. Como o trabalho tem como objetivo a construção de artefatos para aprimorar as relações em um ambiente de agentes inovadores, o processo investigativo deve conjugar esforços teóricos-metodológicos com a utilidade prática identificada para a sociedade.


%A condução de pesquisas requer um rigor científico e modelo a ser seguido para assegurar que sejam implementados os passos sistemáticos para explicar, descrever, explorar ou predizer fenômenos e suas relações. Esta conduta possibilita a replicação das etapas da pesquisa e facilita a condução de experimentos complementares.

Este capítulo apresenta a abordagem metodológica adotada para assegurar o rigor da pesquisa. Segundo \citeonline{dresch:2015}, a seriedade na condução de um trabalho científico pode ser alcançada através da utilização de métodos de pesquisa alinhados com a natureza do problema que se deseja estudar. Tal condição é fundamental para assegurar a validade dos resultados e seu reconhecimento como estudo confiável. A seguir, são apresentadas as razões pelas quais foi escolhido o método de pesquisa \textit{design science research}, a caracterização do objeto de estudo e os passos definidos para o processo investigativo.

\section{Ecossistemas de Startups e a Ciência do Artificial}
A construção de conhecimento científico demanda um processo de desenvolvimento reconhecido pela comunidade acadêmica e utilizado pelos pesquisadores \cite{kuhn:2003}. Esse método científico deve ser escolhido de acordo com as características da área de estudo. No campo de sistemas de informação, a busca por modelos adequados e a caracterização e condução desses procedimentos têm se tornado comuns \cite{hevner-et-al-2004,peffers-et-al:2007,lee-et-al:2014}. 

%A pesquisa no campo de sistemas de informação tem buscado novas bases epistemológicas e métodos de pesquisa que sejam mais adequados à essência de seus problemas. Basicamente, o propósito das ciências naturais é buscar a compreensão de fenômenos complexos do mundo, de como as coisas são, se comportam e interagem, enquanto as pesquisas em sistemas de informação estão focadas na busca por soluções para certos problemas, projetando e criando artefatos que sejam aplicáveis na indústria \cite{dresch:2015}. Proposto por \citeonline{simon:1969}, o paradigma de \textit{design science}, ou ciência do artificial, é o conjunto de conhecimentos que objetivam intervir em situações existentes utilizando artefatos construídos pelo homem visando alcançar melhores resultados com foco na solução de problemas.

Os estudos no campo de sistemas de informação têm buscado novas bases epistemológicas e métodos de pesquisa que sejam mais adequados à essência de seus problemas. Segundo \citeonline{hevner-et-al-2004}, há dois paradigmas complementares e indissociáveis para a produção de conhecimento em sistemas de informação: ciência comportamental e \textit{design science}. A ciência comportamental é um arcabouço teórico utilizado para compreender fenômenos complexos de organizações e das relações pessoais. Essas teorias geram impactos na implementação de interfaces humano-computadores, no conteúdo das informações e nas decisões gerenciais relacionadas a sistemas de informação. O paradigma de \textit{design science}, ou ciência do artificial, é o conjunto de conhecimentos que objetivam intervir em situações existentes utilizando artefatos construídos pelo homem visando alcançar melhores resultados com foco na solução de problemas \cite{simon:1996}.

As pesquisas em sistemas de informação buscam aprofundar conhecimentos de tecnologias da informação em organizações humanas. A ciência comportamental se ocupa de compreender o impacto da tecnologia nas relações humanas ao passo que \textit{design science} foca no desenvolvimento dos artefatos \cite{lee-et-al:2014}. O uso de um artefato de TI em um contexto organizacional deve então ser objeto de estudo em pesquisa de ciência comportamental. Enquanto a pesquisa em \textit{design science} abrange o desenvolvimento e avaliação de artefatos de TI visando resolver problemas organizacionais \cite{hevner-et-al-2004}. A Figura~\ref{arcabouco-pesquisaSI} apresenta o arcabouço de pesquisas em sistemas de informação proposto por \citeonline{hevner-et-al-2004}. Segundo os autores, a ciência comportamental direciona sua pesquisa através do desenvolvimento e justificação de teorias que explicam ou predizem fenômenos relacionados às necessidades do negócio. \textit{Design science} foca na construção e avaliação de artefatos concebidos para lidar com necessidades do negócio.

\begin{figure}
	\centering
	\includegraphics[height=3.3in, width=5.7in]{./figs/framework_pesquisa_hevner}
	\caption{Arcabouço de Pesquisa em Sistemas de Informação \cite{hevner-et-al-2004}.}
	\label{arcabouco-pesquisaSI}
\end{figure}

O presente trabalho está relacionado à utilização de análise de redes sociais para a divulgação de informações acerca da temática de ecossistemas de startups, visando promover a comunidade empreendedora. O objetivo do estudo, a fim de contribuir com o tema, é projetar e desenvolver artefatos e soluções prescritivas, o que  justifica a aplicação do arcabouço teórico de \textit{design science} \cite{dresch:2015}.

\section{Design Science Research}
A pesquisa é um processo investigativo sistemático que busca desenvolver ou refinar teorias, ou mesmo resolver problemas. O método \textit{design science research} fundamenta e operacionaliza a condução de pesquisas utilizando a base epistemológica da \textit{design science}. Como sua abordagem é orientada a problemas, suas etapas visam compreender as necessidades, construir e avaliar artefatos que permitam transformar as situações, alterando suas condições para estados melhores ou desejáveis \cite{hevner-et-al-2004,dresch:2015}.

A metodologia de pesquisa \textit{design science} possui princípios, práticas e procedimentos necessários para conduzir um trabalho científico, assegurando consistência com a literatura anterior, através de etapas definidas para a pesquisa e de um modelo mental para apresentar e avaliar os resultados alcançados \cite{peffers-et-al:2007}. 

No presente trabalho, optou-se pelo processo descrito por \citeonline{vanAken-et-al:2012}, que defende a utilização da \textit{design science} como uma maneira de diminuir a distância entre a academia e as organizações. Seu modelo prevê um ciclo para a resolução de problemas no qual situações particulares devem ser generalizadas para classes de problemas, ilustrado na Figura~\ref{ciclo_resolucao-problema}. A generalização possibilita que o conhecimento gerado seja aproveitado em situações similares. 

\begin{figure}
	\centering
	\includegraphics[height=3.7in, width=5.8in]{./figs/ciclo_resolucao_problema}
	\caption{Ciclo para Resolução de Problemas \cite{vanAken-et-al:2012}.}
	\label{ciclo_resolucao-problema}
\end{figure}

A execução do modelo se inicia com a percepção do problema. Durante essa etapa, o pesquisador deve compreender e caracterizar o problema. A definição do problema será usada para desenvolver um artefato que possa oferecer uma solução. O propósito da solução deve estar claro, como forma de motivar o pesquisador e o público da pesquisa \cite{peffers-et-al:2007}. Para executar essa atividade, será necessário conhecer o estado da arte na área do problema e a importância de sua solução \cite{vanAken-et-al:2012}.

Durante a análise e diagnóstico são verificadas particularidades internas e externas do objeto de estudo, buscando identificar todas as nuances que cercam o problema. Em seguida o pesquisador deve utilizar critérios pré-definidos para caracterizar o problema.

Cercado de todas as informações acerca do problema, a etapa de projeto da solução foca-se na construção de artefatos: determinação das funcionalidades e desenvolvimento que transformarão o conhecimento da teoria em uma abordagem de solução \cite{peffers-et-al:2007}.

No decorrer da etapa de intervenção, a solução proposta é implementada em uma ou mais instâncias do problema. É preciso planejar e escolher adequadamente o método de pesquisa para a intervenção, que pode ser experimentação, simulação, estudo de caso, prova de conceito ou atividade apropriada.

Por fim, a fase de aprendizagem e avaliação leva a uma reflexão acerca dos dados coletados na intervenção e toda experiência adquirida durante o ciclo. A avaliação dos resultados e toda compreensão gerada auxilia no direcionamento dos próximos passos do pesquisador, podendo inclusive levar a identificação de novos problemas a serem estudados, iniciando um novo ciclo \cite{dresch:2015}.


\section{Condução da Pesquisa}

O modelo de pesquisa adotado prevê a execução dos seguintes passos: caracterização das relações em ecossistemas de startups, diagnóstico inicial de parcerias, proposta de ferramentas de apoio, planejamento de estudos experimentais, avaliação e aprendizagem com os resultados. O aprendizado e as experiências produzidas nesse ciclo serão detalhadas e seus apontamentos servirão de base para a generalização do método para torná-lo aplicável a um número maior de comunidades empreendedoras. Os resultados da pesquisa estão sendo utilizados para fomentar novos trabalhos e aprimorar o método e as métricas propostas. 

A caracterização das relações em ecossistemas de startups vem sendo realizada em duas pesquisas complementares – um estudo secundário, conduzido para identificar e avaliar os resultados relevantes encontrados na literatura; e uma revisão bibliográfica acerca de análise de redes sociais, que buscou identificar as técnicas atualmente aplicadas para a caracterização das relações de parceria em comunidades empreendedoras.


Para traçar um diagnóstico da situação e aumentar a conscientização acerca dos ecossistemas empreendedores, estão sendo realizadas algumas entrevistas com os participantes mais relevantes da comunidade de empreendedorismo das incubadoras tecnológicas dos institutos federais. Por exemplo, para caracterizar os tipos de parcerias presentes nestas comunidades, foi conduzida uma pesquisa exploratória de natureza qualitativa na incubadora tecnológica de empresas do CEFET-RJ, a IETEC. Através dos tipos de conexões definidas por \citeonline{motoyama-waltins:2014}, elaborou-se um questionário aberto que foi aplicado com empreendedores e com o gerente da incubadora, conforme detalhado na Tabela~\ref{questoes-ietec}. Com base nos resultados obtidos, foi possível observar importantes indícios da natureza dos relacionamentos entre os diferentes atores da IETEC. Parte dos resultados foram apresentados no Encontro de Inovação em Sistemas de Informação de 2017, em Lavras, Minas Gerais \cite{escalfoni-irineu-oliveira:2017}.

\begin{table}
	\centering
	
	\caption{Questionário para Caracterização das Parcerias na IETEC/CEFET-RJ \cite{escalfoni-irineu-oliveira:2017}}
	\begin{tabular}{|l|l|} 
		\hline
		\cellcolor{lightgray!25}
		&
		\cellcolor{lightgray!25}
		\\
		\cellcolor{lightgray!25}
		\textbf{Tipos de Parcerias} 
		& 
		\cellcolor{lightgray!25}
		\textbf{Questões} 
		\\[6pt]
		\hline
		&
		\\
		\textbf{Conexões entre}
		& (a) Quem são os principais parceiros? \\
		\textbf{Empreendedores:}
		& (b) O que leva a buscar cooperação? \\
		& (c) Como são formadas e mantidas as parcerias? Há \\
		& um processo formal de identificação de colaboradores? \\
		& (d) Que tipo de recurso é compartilhado? \\
		& (e) Que critérios são adotados para aceitação de \\
		& parceiros?
		\\[6pt] 
		\hline
		&
		\\
		\textbf{Conexões entre }
		& (a) Quais as principais motivações para buscar \\
		\textbf{Empreendedores e }
		& apoio de um instituições de suporte? \\
		\textbf{Principais Organizações}
		& (b) Quais são as organizações de fomento mais\\ 
		\textbf{de Apoio:}
		& importantes para o seu negócio? \\
		
		& (c) Como tais órgãos têm auxiliado o seu negócio?\\

		&
		\\[6pt] 
		\hline
		&
		\\
		\textbf{Demais Conexões}
		& (a) Seu negócio mantém vínculos com outras\\ 
		\textbf{de Apoio:}
		& organizações? 
		\\
		
		& (b) Qual é o papel destas parcerias?
		\\[6pt] 
		\hline
		&
		\\
		\textbf{Conexões entre }
		& (a) Possui parcerias com outros órgãos de  \\
		\textbf{Organizações de}
		&apoio (quais)?
		\\
		\textbf{Suporte:}
		& (b) O que se busca neste tipo de parceria? \\

		& (c) Como funcionam tais conexões?		 
		\\[6pt] 
		\hline
	\end{tabular}
	\label{questoes-ietec}
\end{table}

Outros estudos complementares também estão sendo realizados, inclusive com a proposição e experimentação de métodos e ferramentas. A exemplo dessas atividades, foi proposto um método para a identificação de interesses, influenciadores e especialistas em comunidades, a partir de heurísticas propostas sobre métricas de ARS. O exemplo de aplicação da proposta foi realizado em um grupo de conversas formado por gerentes de incubadoras tecnológicas de institutos federais de 17 estados e diferentes regiões do país. Com anuência dos  participantes, foram coletados 4 meses de conversas, com cerca de 1700 postagens, categorizadas em 267 tópicos de discussão. Foi possível medir a repercussão, relevância e nível de interesse dos participantes com relação aos assuntos publicados. Buscou-se também analisar a influência e a reputação atribuída aos participantes através da classificação das mensagens enviadas ao grupo \cite{escalfoni-et-al:2018}.

Em um trabalho seguinte, submetido ao \textit{Workshop on Big Social Data and Urban Computing 2018} (BIDU 2018), foi proposto um método para o mapeamento de características tecnológicas e sociais em comunidades empreendedoras. O objetivo do trabalho era identificar os recursos disponíveis na rede para aumentar a integração e a eficiência das parcerias existentes. O processo proposto é formado pelas etapas de coleta de dados, construção de grafos e classificação de relações e análise e visualização de parcerias. %\cite{mamede_et_al:2018}. 

O experimento conduzido buscou mapear todo o ecossistema empreendedor formado pelas incubadoras e parques tecnológicos vinculados às principais universidades e centros de pesquisa do estado do Rio de Janeiro e seus parceiros. Os dados dos empreendimentos foram coletados a partir da base de dados do ReINC e então, fez-se um cruzamento com os dados sociais dos empreendedores obtidos a partir da mídia social on-line do LinkedIn \footnote{http://www.linkedin.com}. Foram observados 19 incubadoras e parques tecnológicos, que possuíam cerca de 400 empresas, entre projetos,  empreendimentos incubados, graduadas e empresas associadas. Como resultado do estudo, foram identificados os perfis dos empreendedores de cada incubadora ou parque tecnológico, com suas competências, habilidades e experiência em outros empreendimentos e empregos. Também foi possível identificar os interesses e traços de cultura em comum entre os responsáveis por negócios. Tais características podem ser aproveitados em trabalhos futuros de recomendação de parcerias e análise de comunidades de negócio.


A proposta de método da tese é fruto dos estudos conduzidos e das interações com a comunidade. O detalhamento do método e o desenvolvimento da ferramenta estão planejados para serem realizados ao longo deste ano. Para avaliar a proposta, esperamos realizar experimentos a partir das bases de dados disponíveis - em âmbito local, com o ecossistema empreendedor da IETEC/CEFET-RJ e outros mais abrangentes, com a comunidade empreendedora do estado do Rio de  Janeiro, através da base de dados do ReINC, disponível para consultas na Internet.


%Estabelecer os passos da nossa pesquisa baseado nas seguintes etapas:\textbf{Identificação do problema, Conscientização do problema, Revisão sistemática da literatura, Identificação dos artefatos e configuração das classes de problemas, Proposição de artefatos para resolução do problema, Projeto do artefato, Desenvolvimento do artefato, Avaliação do artefato, Explicitação das aprendizagem e conclusão, Generalização para uma classe de problemas e comunicação dos resultados, Aplicação das heurísticas}

%No presente trabalho, adotamos os passos descritos na abordagem proposta por \citeonline{dresch:2015}, de forma a obter resultados generalizáveis e que possam ser reproduzidos em outros contextos.

%A seguir são apresentadas as etapas de trabalho planejadas para esta pesquisa, listando as atividades e estudos experimentais relacionados.


%\subsection{Caracterização das Parcerias em Ecossistemas de Startups}

%Seguindo o modelo proposto por \citeonline{dresch:2015}, esta primeira etapa é caracterizada pela investigação e conscientização do problema. 

%\subsubsection{Estudo Secundário}
%A execução de estudos secundários constitui uma importante ferramenta para avaliar os resultados relevantes anterior. No presente trabalho, está sendo realizada uma revisão sistemática com o objetivo de caracterizar os conceitos e desafios relacionados às parcerias firmadas em ecossistemas de startups. 

%\subsubsection{Revisão de Literatura sobre Análise de Redes Sociais}
%Além dos estudos secundários, está sendo  realizada uma revisão de literatura com o objetivo de identificar meios para caracterizar as relações em ecossistemas de startups.

%\subsubsection{Investigação de Artefatos}
%Corresponde a uma investigação na literatura dos métodos, ferramentas e técnicas para a caracterização das relações em comunidades empreendedoras. 

%\subsection{Proposição de Artefatos}
%A fase de proposição de artefatos prevê uma série de estudos primários para o refinamento da técnica proposta. 

%\subsubsection{Projeto do Artefato}

%\subsubsection{Desenvolvimento do Artefato}

%\subsubsection{Avaliação do Artefato}

%\subsection{Avaliação do Método}


%\section{Cronograma Proposto}

\section{Plano de Trabalho e Cronograma}
As seguintes tarefas serão executadas, de acordo com o cronograma descrito na Tabela~\ref{cronograma}.
\begin{enumerate}
	
	\item \hypertarget{txtask1}{\hyperlink{task1}{Escrita e submissão da revisão sistemática acerca das relações de parceria em ecossistemas de startups.}}
	\item \hypertarget{txtask2}{\hyperlink{task2}{	Fundamentação mais precisa sobre métricas aplicáveis a comunidades empreendedoras.}}
	\item \hypertarget{txtask3}{\hyperlink{task3}{Projeto e Desenvolvimento da Plataforma Coral.}}
	\item \hypertarget{txtask4}{\hyperlink{task4}{Experimento em Âmbito Local.}}
	\item \hypertarget{txtask5}{\hyperlink{task5}{Refinamentos do Método após Experimento Local.}}
	\item \hypertarget{txtask6}{\hyperlink{task6}{Experimento em Âmbito Regional.}}
	\item \hypertarget{txtask7}{\hyperlink{task7}{	Escrita de Artigos com os Resultados Obtidos.}}
	\item \hypertarget{txtask8}{\hyperlink{task8}{	Atualização do Levantamento de Trabalhos Relacionados}}
	\item \hypertarget{txtask9}{\hyperlink{task9}{Escrita da Tese}}
	\item \hypertarget{txtask10}{\hyperlink{task10}{Defesa da Tese}}
	
\end{enumerate}

%aqui os hypertargets precisam ser definidos na linha de cima da tabela para renderizarem certo na maioria dos leitores pdf

\begin{table}[htbp]
	\caption{Planejamento das Atividades a serem Executadas.}
	\begin{tabular}{|c|c|c|c|c|c|c|c|c|c|c|c|c|}
		\hline
		~ & \multicolumn{9}{c|}{\textbf{\textit{2018}}} & \multicolumn{3}{c|}{\textbf{\textit{2019}}}    \\ \cline{2-13}
		~  & \textbf{Abr} & \textbf{Mai} & \textbf{Jun} & \textbf{Jul} & \textbf{Ago}  & \textbf{Set} & \textbf{Out} & \textbf{Nov} & \textbf{Dez} & \textbf{Jan} & \textbf{Fev}  & \textbf{Mar} \hypertarget{task1}{} \\ \hline
		\hyperlink{txtask1}{1}  & X & X & ~   & ~   & ~    & ~   & ~   & ~   & ~   & ~   & ~    & ~   \hypertarget{task2}{}\\ \hline
		\hyperlink{txtask2}{2}  & X  & X & ~ & ~ & ~    & ~   & ~   & ~   & ~   & ~   & ~    & ~   \hypertarget{task3}{}\\ \hline
		\hyperlink{txtask3}{3}  & ~   & X  & X & X  & ~    & ~   & ~   & ~   & ~   & ~   & ~    & ~   \hypertarget{task4}{}\\ \hline
		\hyperlink{txtask4}{4}  & ~   & ~   & ~ & ~ & X  & X & ~   & ~   & ~   & ~   & ~    & ~   \hypertarget{task5}{}\\ \hline
		\hyperlink{txtask5}{5}  & ~   & ~   & ~   & ~   & ~    & ~  & X & ~   & ~   & ~   & ~    & ~   \hypertarget{task6}{}\\ \hline
		\hyperlink{txtask6}{6}  & ~   & ~   & ~   & ~   & ~    & ~   & ~ & X & X & ~   & ~    & ~   \hypertarget{task7}{}\\ \hline
		\hyperlink{txtask7}{7}  & ~   & ~   & ~   & ~   & ~    & X & X & X & X  & ~   & ~    & ~   \hypertarget{task8}{}\\ \hline
		\hyperlink{txtask8}{8}  & ~   & ~   & ~   & ~   & ~    & ~   & ~   & ~   & ~   & X  & ~  & ~   \hypertarget{task9}{}\\ \hline
		\hyperlink{txtask9}{9}  & ~   & ~   & ~   & ~   & ~    & ~   & ~   & ~   & ~   & X & X & ~ 
		\hypertarget{task10}{}\\ \hline
		\hyperlink{txtask10}{10}  & ~   & ~   & ~   & ~   & ~    & ~   & ~   & ~   & ~   & ~ & ~ & X \\
		\hline
	\end{tabular}
	
	\label{cronograma}
\end{table}

\section{Considerações Finais}
Neste capítulo foi apresentada a abordagem metodológica utilizada para a condução do presente trabalho. Em função da natureza do objeto de estudo e da proposta de criação de artefatos computacionais para intervir nas comunidades empreendedoras, optou-se por utilizar uma estratégia de pesquisa baseada no método \textit{Design Science Research}. A metodologia \textit{Design Science Research} é um ferramental para operacionalizar a condução de pesquisas utilizando a base epistemológica da \textit{Design Science}. 

Em seguida, foi definido um modelo de condução da pesquisa baseado em ciclos evolutivos para ampliar a compreensão acerca do objeto de estudo. Dessa forma, os resultados obtidos através da execução do processo foram decorrentes de atividades cíclicas de pesquisa na literatura, entrevistas com especialistas e construção e execução de experimentos.